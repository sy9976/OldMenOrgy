\documentclass[A_4paper,12pt]{article}
\usepackage{polski}
\usepackage{graphicx, xcolor} % Required for the inclusion of images
\usepackage{natbib} % Required to change bibliography style to APA
\usepackage{amsmath} % Required for some math elements 
\usepackage[utf8]{inputenc}
\setlength\parindent{0pt} % Removes all indentation from paragraphs
\usepackage{svg}
\usepackage{geometry}
\usepackage{afterpage}
\usepackage{caption}
\usepackage{layout}
\usepackage{tabularx}
\usepackage{wrapfig}
\usepackage{float}
\usepackage{capt-of}

\renewcommand{\labelenumi}{\alph{enumi}.} % Make numbering in the enumerate environment by 
%\usepackage{times} % Uncomment to use the Times New Roman font

\title{Laboratorium Informatyki w Medycynie \\ 3 punkt kontrolny} % Title
\author{Szymon \textsc{Gramza} 109785  \\ Przemysław \textsc{Hoffmann} 109786} % Author name

\date{23.05.2015r.} % Date for the report

\begin{document}

\maketitle % Insert the title, author and date

\begin{center}
\begin{tabular}{l r}
Prowadzący: & mgr inż. Tomasz Pawlak \\
Temat zadania: & symulator tomografu komputerowego
\end{tabular}
\end{center}

\newpage

\section{Opis problemu}
Tematem projektu zaliczeniowego jest symulator tomografu komputerowego.
Zgodnie z wymaganiami symulator ten powinien pozwala na:
\begin{itemize}
\item akwizycję rzutów obrazów 1D z zadanego obrazu 2D
\item prezentacje użytkownikowi tych rzutów
\item rekonstrukcję obrazów 2D z rzutów 1D przy użyciu odwrotnej transformaty Radona
\item prezentację zrekonstruowanego obrazu
\end{itemize}
Dodatkowo aplikacja pozwala w prosty sposób sterować głównymi parametrami symulatora.

\subsection{Zasada działania}
[definicja]
[budowa]
[schematy]
[zasada działania]
[problemy zagadnienia]
[wprowadzenie ruchu obrotowego]

\subsubsection{Generacje tomografów}
Od czasów wynalezienia tomografu zmieniały koncepcje i usprawnienia mające na celu zwiększenie ich szybkości i wydajności.
W związku z tym powastała poniższa klasyfikacja na generacje:
\begin{itemize}
\item Generacja I - skaner składał się z pojedynczego emitera i detektora. Lampa i detektor wykonywały ruchy translacyjne i rotacyjne.
\item Generacja II - zwiekszono liczbę detektorów co zmniejszyło liczbę ruchów translacyjnych lampy.
\item Generacja III - wyeliminowano ruch translacyjny poprzez rozmieszczenie detektorów na łuku pierścienia obracającego się razem z lampą dookoła pacjenta.
\item Generacja IV - detektory umieszczone zostały na stałe na pierścienu, ruch obrotowy wykonuje tylko lampa.
\end{itemize}


\section{Opis rozwiązania}
Poniżej znajduje się opis użytej technologii oraz lista zastosowanych algorytmów z wyróżnieniem na poszczególne etapy.
\subsubsection{Technologia}
Program zaimplementowany został w języku Python (wersja 2.7) z wykorzystaniem następujących bibliotek:
\begin{itemize}
\item tkinter - tworzenie, obsługa oraz zwalnianie interfejsu graficznego. 
	Pozwala na intuicyjne poruszanie się po aplikacji, wczytywanie oraz zapisywanie przetwarzanych obrazów w szczególności wczytanie obrazu do generacji sinogramu, zapis sinogramu, wczytanie sinogramu, zapis zrekonstruowanego obrazu. Dodatkowo biblioteki użyto do wygodnego sterowania najważniejszymi parametrami działania symulatora.
\item numpy - działania na macierzach, konwersje
\end{itemize}
\newpage
Przyjęto model tomografu o następujących parametrach:
\begin{itemize}
\item alfa - kąt o który obracany jest układ emiterów
\item beta - szerokość wiązki
\item odległość układu emiterów od macierzy obrazu
\item coś jeszcze
\item szerokość filtra
\end{itemize}

\subsubsection{Algorytmy}
W tej sekcji znajduje się opis poszczególnych etapów przetwarzania i użytych w tym celu algorytmów.
W zakresie działania symulatora wyróżniamy głównie przetwarzanie wczytanego obrazu do postaci sinogramu, oraz rekonstrukcję otrzymanego sinogramu do postaci oryginalnej.
\begin{itemize}
\item W przypadku generowanie sinogramu, oryginalny obraz zostaje wczytany w skali szarości a następnie przechowany w pamięci w postaci lokalnej macierzy na której wykonywane są następujące obliczenia.
*Bresenham
*wyznaczanie łuku emiterów
*środki detektorów
*akwizycja


\item Rekonstrukcja obrazu korzysta z wcześniejszych metod, jednak zmienia się kierunek emisji (emisja następuje od macierzy detektorów w kierunku emitera, który staje się odbiornikiem)
*filtr
\end{itemize}


Poniżej znajduje się kilka pojęć, których zrozumienienie i wykorzystanie wymagane jest do rozwiązania projektu.
\subsubsection{Transformata Radona}
W roku 1905 W. Radon udowodnił następujące twierdzenie: „Obraz obiektu dwuwymiarowego można zrekonstruować na podstawie nieskończonej ilości rzutów jednowymiarowych”. Rzutowanie to odpowiada wykonywaniu na obiekcie pewnej transformacji, nazywanej Transformacją Radona.
Po dokonaniu transformacji z otrzymanych wyników otrzymyje się sinogram, czyli wykres będący wizualizacją owych wyników.
Dokonanie na sinogramie Odwrotnej Transformacji Radona umożliwia zrekonstruowanie obrazu obiektu. Zrekonstruowany obraz zawierał będzie pewne zniekształcenia, lecz wraz ze wzrostem liczby projekcji powinien co raz bardziej odzwierciedlać obraz sprzed transformacji.

\subsubsection{Algorytm Bresenham'a}
Służy do wyznaczania pikseli leżących na odcinku między dwoma punktami. W naszym zastosowaniu służy do wyznaczania pikseli obrazu, przez które przechodzi promień oraz wyznaczenia wartości pochłoniętego promieniowania.

\section{Testy}
Przeprowadzenie testów i jakaś funkcja oceny

\section{Wnioski}
Na podstawie testów, ocena implementacji, jakie parametry optymalne w funkcji np. rozmiaru obrazu, albo ogólnie

\section{Ulepszenia}
Na podstawie wniosków co można usprawnić, co było nie tak, co może być lepsze

\section{Bibliografia}
\bibliographystyle{apalike}
\bibliography{sample}
\end{document}